% An example using the LaTeX class.
\DocumentMetadata
{
    lang = en-US,
    pdfversion = 2.0,
    pdfstandard = a-4f,
    pdfstandard = x-6,
    pdfstandard = ua-2,
    testphase ={
        phase-III,
        table,
        title,
        math,
        firstaid,
    },
    xmp = true,
}
\documentclass[9pt]{ametsocextabs}
\title{Using the AMS Extended Abstract LaTeX Class and Template}
\papernumber{\hfill}
\author{Chris Slocum\aff{a}\thanks{\textit{Previous affiliation}: Department of Atmospheric Science, Colorado State University, Fort Collins, Colorado.}\correspondingauthor{Chris Slocum, \url{https://github.com/CSlocumWX/}}
}
\affiliation{\aff{a}{Fort Collins, Colorado}}

\begin{document}
\maketitle
\section{Introduction}
The purpose of the \texttt{ametsocextabs} \LaTeX\ class
template is to assist presenters at American Meteorological Society
(AMS) conferences prepare an extended abstract that follows the AMS
Extended Abstract Instructions \citep{AMS2025abs}. While AMS no
longer provides hard copies of the extended abstracts, the abstracts
allow presenters to capture their presentation in greater detail.

Also, The Max A. Eaton Student Prize, awarded for an outstanding
student paper presented at each conference on hurricanes and tropical
meteorology, requires students to submit an extended abstract and
evaluates the quality of both the extended abstract and presentation
\citep{AMS2025student, AMS2025Eaton}.

\section{The \lowercase{\texttt{ametsocextabs}} \LaTeX\ Class}

\subsection{Font Size}

The AMS Extended Abstract Instructions permits font sizes of 9--10~pt
in a sans-serif typeface such as Helvetica \citep{AMS2025abs}.
To select a size, include one of the following options in the
documentclass call:
\begin{itemize}
    \item \texttt{9pt} for a 9~pt font size (the default), and
    \item \texttt{10pt} for a 10~pt font size.
\end{itemize}
As an example, using the 9~pt option would look like
\texttt{\textbackslash documentclass[9pt]\{ametsocextabs\}}.

\subsection{Mathematical Formula Typeface}
Mathematical Formulas should follow the AMS author guidelines
\citep{AMS2025math}. One difference between AMS journal articles
and extended abstracts is the typeface in which the mathematical
formulates are presented.

Here, by default, mathematical formulas render with a sans-serif
typeface to stay consistent with the AMS guidelines shown by this
example for the cosine function:
\begin{equation}
    y = \cos (x),
    \label{eq:equation}
\end{equation}
where $x$ is the independent variable and $y$ is the dependent.
Equations can be referenced in the usual way, as with \eqref{eq:equation}.

However, users can add the \texttt{serif} class option to the
documentclass call to change the mathematical formulas from a
sans-serif typeface to a serifed typeface (e.g.,
\texttt{\textbackslash documentclass[serif]\{ametsocextabs\}}). This
option might be useful when symbols are not clear (e.g., a lowercase
`l' renders as $l$ and an uppercase 'i' as $I$).

\subsection{Citations and References}
The \lowercase{\texttt{ametsocextabs}} \LaTeX\ class template
includes the AMS BibTeX style file \texttt{ametsocV6}. With which,
presenters can follow the citation instructions included in the
documentation for the AMS \LaTeX\ files \cite{AMS2025latex, AMS2025doc}.

The AMS BibTeX style uses two basic citation macro commands:
\begin{enumerate}
    \item \texttt{\textbackslash citet} for textual citations \rightarrow \citet{Eliassen1951}, and
    \item \texttt{\textbackslash citep} for parenthetical citations  \rightarrow \citep{Eliassen1951}.
\end{enumerate}
You can add text to a parenthetical citation and multiple
citations just as in the AMS \LaTeX\ files
\citep[e.g.,][]{Eliassen1951,AMS2025latex, AMS2025doc}.

Store references in a \texttt{.bib} bibliography file such as the
provides \texttt{references.bib} file. Entries should follow AMS's
style with the appropriately populated fields (e.g., AMS does not
use issue but BibTeX will still render it). See the ``How to Use the
American Meteorological Society Bibliographic Style File'' PDF file
included in the AMS \LaTeX\ files \cite{AMS2025latex, AMS2025doc}.

\subsection{Sections}
The class provides multilevel or nested sections that provide
a section depth of four levels:
\begin{itemize}
    \item \texttt{\textbackslash section},
    \item \texttt{\textbackslash subsection},
    \item \texttt{\textbackslash subsubsection}, and
    \item \texttt{\textbackslash paragraph}.
\end{itemize}
For tagged PDFs, make sure that the nesting is used in the
appropriate order and not for stylistic display.

The class contains two additional, optional sections that should
only appear once in the document:
\begin{itemize}
    \item \texttt{\textbackslash datastatement} ---
        provides a place where presenters can note where readers will
        find the data used in the research, and
    \item \texttt{\textbackslash acknowledgments} ---
        provides a place where presenters can highlight funding sources
        and any disclaimers associated with the work.
\end{itemize}

The class also has the ability to add appendixes. See the Appendix
for more information.

\subsection{Tables and Figures}
Nothing special here. Tables and figures can be added as one would
normally do. As a reminder, AMS format is to have the caption above
a table (see Table~\ref{tab:cos}) and below a figure. Both can be one
or two columns wide.

\begin{table}
    \caption{Values of $y$ given $x$ using the cosine function in \eqref{eq:equation}.}\label{tab:cos}
    \begin{center}
        \tagpdfsetup{table-header-rows={1}}
        \begin{tabular}{rr}
            \hline\hline
            $x$ & $y$\\
            \hline
            $0$ & $1$ \\
            $\pi/2$ & $0$ \\
            $\pi$ & $-1$ \\
            \hline
        \end{tabular}
    \end{center}
\end{table}


%%%%%%%%%%%%%%%%%%%%%%%%%%%%%%%%%%%%%%%%%%%%%%%%%%%%%%%%%%%%%%%%%%%%%
% DATA AVAILABILITY STATEMENT
%%%%%%%%%%%%%%%%%%%%%%%%%%%%%%%%%%%%%%%%%%%%%%%%%%%%%%%%%%%%%%%%%%%%%
% The data availability statement is where authors should describe
% how the data underlying the findings within the article can be
% accessed and reused.  Authors should attempt to provide
% unrestricted access to all data and materials underlying reported
% findings. If data access is restricted, authors must mention this
% in the statement.
%
\datastatement The LaTeX class file used to generate this PDF issue
available under a BSD 3-clause license at
\url{https://github.com/CSlocumWX/ams_extended_abstract}.

%%%%%%%%%%%%%%%%%%%%%%%%%%%%%%%%%%%%%%%%%%%%%%%%%%%%%%%%%%%%%%%%%%%%%
% ACKNOWLEDGMENTS
%%%%%%%%%%%%%%%%%%%%%%%%%%%%%%%%%%%%%%%%%%%%%%%%%%%%%%%%%%%%%%%%%%%%%
%
\acknowledgments
The author would like to acknowledge Prof.\ Wayne H.\ Schubert and
Rick Taft for comments on initial versions of the AMS extended
abstract \LaTeX\ class. The author developed the class while a
graduate student in the Schubert Research Group,
Department of Atmospheric Science, Colorado State University. During
the development of this class, the author received funding through
National Oceanographic Partnership Program through ONR contract
N000014-10-1-0145, National Oceanic and Atmospheric Administration
(NOAA) Grants NA090AR4320074, NA14OAR4320125 at the Cooperative
Institute for Research in the Atmosphere at Colorado State
University and through the National Science Foundation under Grants
ATM-0425247, ATM-0837932, AGS-1147120, AGS-1250966, AGS-1546610, and
AGS-1601623.

\appendix
\appendixtitle{An Appendix on Appendixes}
This class uses the AMS format to allow for multiple appendixes.
Whether one or more appendixes, each should have an appendix title
that is added with the \texttt{\textbackslash appendixtitle}
command.

If more than one appendix, a letter should be added at the end of
the appendix command (e.g., \texttt{\textbackslash appendix[A]},
\texttt{\textbackslash appendix[B]}). See the \citet{AMS2025latex}
and \citet{AMS2025doc} for more information.

%%%%%%%%%%%%%%%%%%%%%%%%%%%%%%%%%%%%%%%%%%%%%%%%%%%%%%%%%%%%%%%%%%%%%
%   No need to edit the reference section below
%%%%%%%%%%%%%%%%%%%%%%%%%%%%%%%%%%%%%%%%%%%%%%%%%%%%%%%%%%%%%%%%%%%%%
\bibliographystyle{ametsocV6}
\bibliography{references}
\end{document}
