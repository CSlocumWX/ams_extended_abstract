% Attempt at an AMS extended abstract template
% Developed by Chris Slocum, CSU
% Based off of the AMS LaTeX template
%%%%%%%%%%%%%%%%%%%%%%%%%%%%%%%%%%%%%%%%%%%%%%%%%%%%%%%%%%%%%%%%%%%%%
% DocumentMetadata options
%%%%%%%%%%%%%%%%%%%%%%%%%%%%%%%%%%%%%%%%%%%%%%%%%%%%%%%%%%%%%%%%%%%%%
% DocumentMetadata is new to LaTeX. It increases accessibility and
% helps PDF files meet PDF standards. Note that uses these does not
% guarentee standard compliant PDF files. And, some features are still
% experimental and might not work with your files.
% References:
%   https://latex3.github.io/tagging-project/
%   https://mirrors.mit.edu/CTAN/macros/latex/required/latex-lab/documentmetadata-support-doc.pdf
%   https://mirrors.mit.edu/CTAN/macros/latex/required/latex-lab/latex-lab-table.pdf
% . https://latex3.github.io/tagging-project/tagging-status/
\DocumentMetadata
{
    lang = en-US,
    pdfversion = 2.0,
    pdfstandard = a-4f,
    pdfstandard = x-6,
    pdfstandard = ua-2,
    testphase ={
        phase-III,
        table,
        title,
        math,
        firstaid,
    },
    xmp = true,
}
\documentclass[9pt]{ametsocextabs}
%%%%%%%%%%%%%%%%%%%%%%%%%%%%%%%%%%%%%%%%%%%%%%%%%%
% Enter your paper's/talk's title
\title{Paper title}
% Your paper number is the <session number>.<paper number>
\papernumber{J1A.1}
\author{John Doe\aff{a}\correspondingauthor{John Doe, Street Address, City, AB ZIP code;
        e-mail: \href{mailto:}{John.Doe@affiliation.domain}}
        \and Jane Doe\aff{b}
}
\affiliation{\aff{a}{Affiliation},
             \aff{b}{Affiliation}}

\begin{document}
% This creates the title and corresponding author information.
% This should not have to change or be removed.
\maketitle
%%%%%%%%%%%%%%%%%%%%%%%%%%%%%%%%%%%%%%%%%%%%%%%%%%%%%%%%%%%%%%%%%%%%%
%   PLACE TEXT OF EXTENDED ABSTRACT HERE
%%%%%%%%%%%%%%%%%%%%%%%%%%%%%%%%%%%%%%%%%%%%%%%%%%%%%%%%%%%%%%%%%%%%%
\section{Introduction}
Add your text here.

You can use add inline citations using the same format as with the
American Meteorological Society manuscript \LaTeX template
\citep[e.g.,][]{Eliassen1951}.

In this template, math stays in a serif font as shown by this
example for the cosine:
\begin{equation}
    y = \cos (x),
\end{equation}
where $x$ is the independent variable and $y$ is the dependent.

%%%%%%%%%%%%%%%%%%%%%%%%%%%%%%%%%%%%%%%%%%%%%%%%%%%%%%%%%%%%%%%%%%%%%
% DATA AVAILABILITY STATEMENT
%%%%%%%%%%%%%%%%%%%%%%%%%%%%%%%%%%%%%%%%%%%%%%%%%%%%%%%%%%%%%%%%%%%%%
% The data availability statement is where authors should describe
% how the data underlying the findings within the article can be
% accessed and reused.  Authors should attempt to provide
% unrestricted access to all data and materials underlying reported
% findings. If data access is restricted, authors must mention this
% in the statement.
%
\datastatement
Add Data availability statement here.

%%%%%%%%%%%%%%%%%%%%%%%%%%%%%%%%%%%%%%%%%%%%%%%%%%%%%%%%%%%%%%%%%%%%%
% ACKNOWLEDGMENTS
%%%%%%%%%%%%%%%%%%%%%%%%%%%%%%%%%%%%%%%%%%%%%%%%%%%%%%%%%%%%%%%%%%%%%
%
\acknowledgments
Add acknowledgments here.

%%%%%%%%%%%%%%%%%%%%%%%%%%%%%%%%%%%%%%%%%%%%%%%%%%%%%%%%%%%%%%%%%%%%%
%   No need to edit the reference section below
%%%%%%%%%%%%%%%%%%%%%%%%%%%%%%%%%%%%%%%%%%%%%%%%%%%%%%%%%%%%%%%%%%%%%
\bibliographystyle{ametsocV6}
\bibliography{references}
\end{document}
