% Attempt at an AMS extended abstract template
% Developed by Chris Slocum, CSU
% Based off of the AMS LaTeX template
%%%%%%%%%%%%%%%%%%%%%%%%%%%%%%%%%%%%%%%%%%%%%%%%%%%%%%%%%%%%%%%%%%%%%
% DocumentMetadata options
%%%%%%%%%%%%%%%%%%%%%%%%%%%%%%%%%%%%%%%%%%%%%%%%%%%%%%%%%%%%%%%%%%%%%
% DocumentMetadata is new to LaTeX. It increases accessibility and
% helps PDF files meet PDF standards. Note that uses these does not
% guarentee standard compliant PDF files. And, some features are still
% experimental and might not work with your files.
% References:
%   https://latex3.github.io/tagging-project/
%   https://mirrors.mit.edu/CTAN/macros/latex/required/latex-lab/documentmetadata-support-doc.pdf
%   https://mirrors.mit.edu/CTAN/macros/latex/required/latex-lab/latex-lab-table.pdf
% . https://latex3.github.io/tagging-project/tagging-status/
\DocumentMetadata
{
    lang = en-US,
    pdfversion = 2.0,
    pdfstandard = a-4f,
    pdfstandard = x-6,
    pdfstandard = ua-2,
    testphase ={
        phase-III,
        table,
        title,
        math,
        firstaid,
    },
    xmp = true,
}
\documentclass[9pt]{ametsocextabs}
%%%%%%%%%%%%%%%%%%%%%%%%%%%%%%%%%%%%%%%%%%%%%%%%%%
% Enter your paper's/talk's title
\title{Paper title}
% Your paper number is the <session number>.<paper number>
\papernumber{J1A.1}
\author{John Doe\aff{a}\correspondingauthor{John Doe, Street Address, City, AB ZIP code;
        e-mail: \href{mailto:}{John.Doe@affiliation.domain}}
        \and Jane Doe\aff{b}
}
\affiliation{\aff{a}{Affiliation},
             \aff{b}{Affiliation}}

\begin{document}
% This creates the title and corresponding author information.
% This should not have to change or be removed.
\maketitle
%%%%%%%%%%%%%%%%%%%%%%%%%%%%%%%%%%%%%%%%%%%%%%%%%%%%%%%%%%%%%%%%%%%%%
%   PLACE TEXT OF EXTENDED ABSTRACT HERE
%%%%%%%%%%%%%%%%%%%%%%%%%%%%%%%%%%%%%%%%%%%%%%%%%%%%%%%%%%%%%%%%%%%%%
\section{Introduction}
The purpose of the \lowercase{\texttt{ametsocextabs}} \LaTeX class
template is to assist presenters at American Meteorological Society
(AMS) conferences prepare an extended abstract that follows the AMS
Extended Abstract Instructions \citep{AMS2025abs}. While AMS no
longer provides hard copies of the extended abstracts, the abstracts
allow presenters to capture their presentation in greater detail.

Also, The Max A. Eaton Student Prize, awarded for an outstanding
student paper presented at each conference on hurricanes and tropical
meteorology, requires students to submit an extended abstract and
evaluates the quality of both extended abstract and presentation
\citep{AMS2025student, AMS2025Eaton}.

\section{The \lowercase{\texttt{ametsocextabs}} \LaTeX Class}

\subsection{Font Size}

The AMS Extended Abstract Instructions permits font sizes of 9--10~pt
in a sans-serif typeface such as Helvetica \citep{AMS2025abs}.
To select a size, include one of the following options in the
documentclass call:
\begin{enumerate}
    \item \texttt{9pt} for a 9~pt font size (the default), and
    \item \texttt{10pt} for a 10~pt font size.
\end{enumerate}
As an example, using the 9~pt option would look like
\texttt{\textbackslash documentclass[9pt]\{ametsocextabs\}}.

\subsection{Mathematical Formula Typeface}
Mathematical Formulas should follow the AMS author guidelines
\citep{AMS2025math}. One difference between AMS journal articles
and extended abstracts is the typeface in which the mathematical
formulates are presented.

Here, by default, mathematical formulas render with a sans-serif
typeface to stay consistent with the AMS guidelines shown by this
example for the cosine function:
\begin{equation}
    y = \cos (x),
    \label{eq:equation}
\end{equation}
where $x$ is the independent variable and $y$ is the dependent.
Equations can be referenced in the usual way as with \eqref{eq:equation}.

However, users can add the \texttt{serif} class option to the
documentclass call to change the mathematical formulas from a
sans-serif typeface to a serifed typeface (e.g.,
\texttt{\textbackslash documentclass[serif]\{ametsocextabs\}}). This
option might be useful when symbols are not clear (e.g., a lowercase
`l' renders as $l$ and an uppercase 'i' as $I$).

\subsection{Citations and References}
The \lowercase{\texttt{ametsocextabs}} \LaTeX class template
includes the AMS BibTeX style file \texttt{ametsocV6}. With which,
presenters can follow the citation instructions included in the
documentation for the AMS \LaTeX files \cite{AMS2025latex, AMS2025doc}.

The AMS BibTeX style uses two basic citation macro commands:
\begin{itemize}
    \item \texttt{\textbackslash citet} for textual citations \rightarrow \citet{Eliassen1951}, and
    \item \texttt{\textbackslash citep} for parenthetical citations  \rightarrow \citep{Eliassen1951}.
\end{itemize}
You can add text to a parenthetical citation and multiple
citations just as in the AMS \LaTeX files
\citep[e.g.,][]{Eliassen1951,AMS2025latex, AMS2025doc}.

Store references in a \texttt{.bib} bibliography file such as the
provides \texttt{references.bib} file. Entries should follow AMS's
style with the appropriately populated fields (e.g., AMS does not
use issue but BibTeX will still render it). See the ``How to Use the
American Meteorological Society Bibliographic Style File'' PDF file
included in the AMS \LaTeX files \cite{AMS2025latex, AMS2025doc}.

%%%%%%%%%%%%%%%%%%%%%%%%%%%%%%%%%%%%%%%%%%%%%%%%%%%%%%%%%%%%%%%%%%%%%
% DATA AVAILABILITY STATEMENT
%%%%%%%%%%%%%%%%%%%%%%%%%%%%%%%%%%%%%%%%%%%%%%%%%%%%%%%%%%%%%%%%%%%%%
% The data availability statement is where authors should describe
% how the data underlying the findings within the article can be
% accessed and reused.  Authors should attempt to provide
% unrestricted access to all data and materials underlying reported
% findings. If data access is restricted, authors must mention this
% in the statement.
%
\datastatement
Add Data availability statement here. Use the
\texttt{\textbackslash url\{\}} macro command to link to data.

%%%%%%%%%%%%%%%%%%%%%%%%%%%%%%%%%%%%%%%%%%%%%%%%%%%%%%%%%%%%%%%%%%%%%
% ACKNOWLEDGMENTS
%%%%%%%%%%%%%%%%%%%%%%%%%%%%%%%%%%%%%%%%%%%%%%%%%%%%%%%%%%%%%%%%%%%%%
%
\acknowledgments
Add acknowledgments here.

%%%%%%%%%%%%%%%%%%%%%%%%%%%%%%%%%%%%%%%%%%%%%%%%%%%%%%%%%%%%%%%%%%%%%
%   No need to edit the reference section below
%%%%%%%%%%%%%%%%%%%%%%%%%%%%%%%%%%%%%%%%%%%%%%%%%%%%%%%%%%%%%%%%%%%%%
\bibliographystyle{ametsocV6}
\bibliography{references}
\end{document}
